% XeLaTeX example-single.tex

\documentclass{exam-zh}

\usepackage{siunitx}

\examsetup{
  page/size=a4paper,
  paren/show-paren=true,
  paren/show-answer=true,
  fillin/show-answer=false,
  solution/show-solution=show-stay,
  solution/label-indentation=false
}

\everymath{\displaystyle}

\title{题目}

\begin{document}

% 题目.
\begin{problem}
  已知双曲线$C:x^2-y^2=\lambda(\lambda>0)$,焦点F到其中一条渐近线的距离为$\sqrt{3}$,
  \begin{enumerate}
    \item 求 $ \lambda $ 
    \begin{solution}
      (略)$\lambda = 3$ 
    \end{solution}

    \item 动点$M$,$N$在曲线C上,已知点A(2,-1),直线AM,AN分别与y轴相交的两点关于原点对称,点Q在直线MN上,$AQ\perp MN$,证明:存在定点T,使得$|QT|$为定值,
    \begin{solution}
\\      由(1)知,双曲线方程为$x^2-y^2=3$,容易观察到,点A(2,-1)在双曲线上,
\\     若$MN$的斜率存在,则$MN$的斜率为$k$,令M的坐标为$(x_1,y_1)$,N的坐标为$(x_2,y_2)$,
\\     令MN方程为$y=kx+m$,联立$x^2-y^2=3$,消去$y$,得$x^2-(kx+m)^2=3$,
\\     整理得$(1-k^2)x^2-2kmx-m^2-3=0$,
\\     由韦达定理,$x_1+x_2=\frac{2km}{1-k^2}$,$x_1x_2=\frac{-m^2-3}{1-k^2}$,
\\     由题意,直线AM,AN分别与y轴相交的两点关于原点对称,
\\     令AM在y轴上的交点为$M_1$,AN在y轴上的交点为$N_1$,
\\      则AM方程为$y+1=\frac{y_1+1}{x_1-2}(x-2)$,
\\      令$x=0$,得$M_1$的坐标为$(0,\frac{(-2)(y_1+1)}{x_1-2}-1)$整理为$(0,\frac{2y_1+x_1}{2-x_1})$,
\\      同理,$N_1$的坐标为$(0,\frac{2y_2+x_2}{2-x_2})$,
\\      由题意,$M_1$,$N_1$关于原点对称,
\\      所以$\frac{2y_1+x_1}{2-x_1}+\frac{2y_2+x_2}{2-x_2}=0$,
\\      那么$\frac{2(kx_1+m)+x_1}{2-x_1}+\frac{2(kx_2+m)+x_2}{2-x_2}=0$,
\\     整理得$\frac{(2k+1)x_1+2m}{2-x_1}+\frac{(2k+1)x_2+2m}{2-x_2}=0$
\\      整理得$(2k+1)(x_1x_2)-(2k-m+1)(x_1+x_2)-4m=0$
\\       代入$x_1+x_2$,$x_1x_2$,得$\frac{(2k+1)(-m^2-3)-(2k-m+1)\frac{2km}{1-k^2}-4m}{k^2-1}=0$,
\\     整理得$m^2+4km+6k+4m+3=0$,因式分解得$(m+2k+1)(m+3)=0$,
\\      解得$m=-2k-1$或$m=-3$,显然当$M=-2k-1$时,MN方程为$y=k(x-2)-1$恒过点A(2,-1),不符合题意,
\\      所以$m=-3$,此时MN方程为$y=kx-3$,恒过点$(0,-3)$,
\\      由圆的性质,点Q在以AB为直径的圆上,圆心为AB的中点,半径为AB的一半,那么圆心为$(1,-2)$,半径为$\sqrt{2}$,
\\      所以点Q的轨迹方程为$(x-1)^2+(y+2)^2=2$,且T的坐标为$(1,-2)$,$|QT|=\sqrt{2}$
\\      所以存在定点$T$,使得$|QT|$为定值

    \end{solution}
  \end{enumerate}
\end{problem}
\end{document}